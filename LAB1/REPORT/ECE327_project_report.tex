\documentclass[paper=a4, fontsize=11pt,twoside]{scrartcl} 

\usepackage[a4paper,pdftex]{geometry}	% A4paper margins
\setlength{\oddsidemargin}{5mm}			% Remove 'twosided' indentation
\setlength{\evensidemargin}{5mm}

\usepackage[english]{babel}
\usepackage[protrusion=true,expansion=true]{microtype}	
\usepackage{amsmath,amsfonts,amsthm,amssymb}
\usepackage{graphicx}

% --------------------------------------------------------------------
% Definitions (do not change this)
% --------------------------------------------------------------------
\newcommand{\HRule}[1]{\rule{\linewidth}{#1}} 	% Horizontal rule

\makeatletter							% Title
\def\printtitle{%						
	{\centering \@title\par}}
\makeatother									

\makeatletter							% Author
\def\printauthor{%					
	{\centering \large \@author}}				
\makeatother					

% --------------------------------------------------------------------
% Metadata (Change this)
% --------------------------------------------------------------------
\title{	\normalsize \textsc{Lab 1, Group 1: Basic Gates} 	% Subtitle
	\\[2.0cm]								% 2cm spacing
	\HRule{0.5pt} \\						% Upper rule
	\LARGE \textbf{\uppercase{character storage and display}}	% Title
	\HRule{2pt} \\ [0.5cm]		% Lower rule + 0.5cm spacing
	\normalsize \today			% Todays date
}

\author{
	Richard Carter\\	
	Clemson University\\	
	Department of Electrical and Computer Engineering\\
	\texttt{rcarte4@clemson.edu} \\
}


\begin{document}
% ------------------------------------------------------------------------------
% Maketitle
% ------------------------------------------------------------------------------
\thispagestyle{empty}		% Remove page numbering on this page

\printtitle					% Print the title data as defined above
\vfill
\printauthor				% Print the author data as defined above
\newpage


% ------------------------------------------------------------------------------
% Begin document
% ------------------------------------------------------------------------------

\setcounter{page}{1}		% Set page numbering to begin on this page
\section{Introduction}

This lab required us to construct a system of memory modules, multiplexers, and decoders to create a character storage and display system that also includes a rotate feature. This was accomplished by creating individual parts and using them to construct a larger system. The system was all functional in the end, with the exception of parts of the rotation feature.

\section{Design}\label{design}
\subsection{Memory}
The lowest level components are the d-type flip flops. These are used to store character codes that can be chosen by the user. For the final product, there are generated in a group of 4 in order to store 4 of the 3-bit character codes. Their values are referenced by a 12-bit long signal vector that is fed into the multiple multiplexers.

\subsection{Character Decoder}
The character code chosen by the 5-by-1 multiplexer of the respective display unit is fed into a 7-segment decoder. This feeds a signal representative of a specific character into the 7-segment display. The characters are hard-coded into the hardware and cannot be changed by the user. The default character set is 't', 'i', 'g', 'e', 'r', ' ', but the source code contains another character set in order to display the word "HELLO".

\section{Results}\label{result}
Overall, the lab was a success. The shifting functionality has a bug where left-trailing characters do not properly update, and become stuck to the character value of the first memory module. This is most likely due to a generate statement for the multiple multiplexers having undesired logic. 

\bibliographystyle{abbrv}
\bibliography{myreference}

\section{Appendix}\label{appendix}

\end{document}
This is never printed